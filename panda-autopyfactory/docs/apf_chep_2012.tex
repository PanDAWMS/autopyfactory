\documentclass[a4paper]{jpconf}
%\usepackage{graphicx}
\begin{document}
\title{AutoPyFactory: A Scalable Flexible Pilot Factory Implementation}

\author{J. Caballero$^1$, J. Hover $^1$, P. Love$^2$, G. Stewart$^3$}

\address{$^1$ Brookhaven National Laboratory, PO BOX 5000 Upton, NY 11973, USA}
\address{$^2$ Department of Physics, Lancaster University, Lancaster, LA1 4YB, UK }
\address{$^3$ Department of Physics and Astronomy, University of Glasgow, Glasgow G12 8QQ, UK}

\ead{jcaballero@bnl.gov}

\begin{abstract}
The ATLAS experiment at the CERN LHC is one of the largest users of grid computing
infrastructure, which is a central part of the experiment's computing operations.
Considerable efforts have been made to use grid technology in the most efficient
and effective way, including the use of a pilot job based workload management framework.
In this model the experiment submits 'pilot' jobs to sites without payload. When these
jobs begin to run they contact a central service to pick-up a real payload to execute.
The first generation of pilot factories were usually specific to a single VO, and were
bound to the particular architecture of that VO's distributed processing. A second
generation provides factories which are more flexible, not tied to any particular VO,
and provide new or improved features such as monitoring, logging, profiling, etc.
In this paper we describe this key part of the ATLAS pilot architecture, a second
generation pilot factory, AutoPyFactory.
AutoPyFactory has a modular design and is highly configurable. It is able to send
different types of pilots to sites and exploit different submission mechanisms and queue
characteristics. It is tightly integrated with the PanDA job submission framework,
coupling pilot flow to the amount of work the site has to run. It gathers information
from many sources in order to correctly configure itself for a site, and its decision logic
can easily be updated.
Integrated into AutoPyFactory is a flexible system for delivering both generic and
specific job wrappers which can perform many useful actions before starting to run
end-user scientific applications, e.g. validation of the middleware, node profiling
and diagnostics, and monitoring.
AutoPyFactory now also has a robust monitoring system and we show how this has helped
establish a reliable pilot factory service for ATLAS.
\end{abstract}


\section{Introduction}

\subsection{}

\section{Architecture}

\section{Monitor}


%%\section{Acknowledgements}
\ack{
The authors would like to thank 
}

\section*{References}
\begin{thebibliography}{99}


\item ATLAS Collaboration 1994 ATLAS Technical Proposal 
      {\it CERN/LHCC/94-43} 

\end{thebibliography}

~

%%Notice:
%%This manuscript has been authored by employees of Brookhaven Science Associates, 
%%LLC under Contract No. XXXXXXXXXX with the U.S. Department of Energy. 
%%The publisher by accepting the manuscript for publication acknowledges 
%%that the United States Government retains a non-exclusive, paid-up, irrevocable, 
%%world-wide license to publish or reproduce the published form of this manuscript, 
%%or allow others to do so, for United States Government purposes.

\end{document}

